\section{Discussion}

\subsection{Overarching discussion of papers}
\subsubsection{Depth of Gamification}
Among the four papers that were reviewed, I would say that the first~\cite{panagiotis2016climbing} managed to create the deepest and most interesting gamified experience despite its failings. This is primarily due to the amount of mechanics that were brought into the course through \emph{Kahoot!}, \emph{WWTBAM?} and \emph{Codecademy}. The variety of having three different solutions integrated into the course should help against staleness which otherwise could have occurred if only one was used. 

The paper that handled gamification in the shallowest manner was the third~\cite{ortiz2017gamification} as it essentially only used one mechanic which by itself really does not add anything interesting to the course. It is really hard to justify the use of the word gamification or even pointsification for this level of shallowness as there is little to nothing that could create an interesting experience from the singular use of badges. 

In general, it is also disappointing to see that some of the papers assume that simply taking some elements from games or making a serious game automatically engages the player, regardless of quality. This might be the result of a lack of understanding for games and game design which may point towards the fact that the gamifications and serious games were made by academics who really do not care that much for games. It would definitely have been interesting to see more examples where an actual game developer tried to gamify a course by taking some of the various ideas presented in the different papers and see what the end result would be. 

\subsubsection{Areas that could see improvements}
In general, there are two primary areas that this sample of papers could improve on. This includes relationship towards failure and gamification in relation to player types. 
While there is a lot that can be done on a course level, it is more challenging to change the fact that a course is part of a traditional academic system where courses generally result in a final grade. This academic system is thus directly against the essence of games that allows for infinite and instant retries. Because of this, the students only have one or limited attempts at ''getting it right'' when it comes to the final grade in a course. 

This draws game design elements like risk vs. reward to an extreme as the students cannot retry and experiment until they find a comfortable balance or to optimise their performance. Without the support for instant and infinite retries, it is not possible to allow for game-specific activities like speedrunning where highly skilled players use their skills to achieve the highest scores and shortest play times. The process of speedrunning is an example of optimisation of performance through infinite retries in a competitive setting. This allows high-end players to further improve their skills as long as they already find the base activity engaging enough. Supporting failure and allowing retries is also important for the lower end of the scale as players can reflect upon how their performance resulted in failure and think about what they would do to change the next attempt. While an argument could be made that this already is possible in some capacity in-between university assignments, each assignment still often contributes to the final grade. The fact that a retry is always available can help struggling players to actually keep playing without fully giving up. While all of this could be handled on the course level and create intricate and interesting gamified experiences, it is ultimately limited by the presence of a final grade which oftentimes is handled through an exam and is unchangeable after the examination finishes. 


Furthermore, there is limited thought placed into the usage of game mechanics in relation to the present player types in the various courses. This can result in very hit or miss gamifications that may work well with some students, but not with others. The first paper that was reviewed~\cite{panagiotis2016climbing} seemed to ''hit'' well with the students through the usage of a public and competitive leaderboard. Despite this, it may not have worked so well with another sample of students or in a different course entirely. While it may not exactly be easy to create a gamified experience that is catered to the very specific player types that are present in a sample it is possible to circumvent the problem somewhat by offering more choice. This was handled well in the final paper that was reviewed~\cite{mozelius2014game} as all the competitive aspects were purely optional. By providing enough choice, it should be possible to reach as many different player types as possible to provide them with an experience that they enjoy. The primary problem with choice is that the more room for choice and options that is provided, the more complexity there is to implement and design the experience. At a base level though, it would be nice to see that gamified courses do not implement gamification in a way that excludes specific types of players that may be present.