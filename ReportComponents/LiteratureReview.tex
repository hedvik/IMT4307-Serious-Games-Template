\section{Results: Literature review}
\subsection{Analysis of ''Climbing up the leaderboard''}
\subsubsection{Shallow/Deep Gamification Analysis}
\paragraph{Use of ''surface level'' game mechanics}
In terms of gamification, this study attempted to gamify an introductory programming class using \emph{Kahoot!}, a modified version of \emph{Who Wants To Be a Millionaire?}(\emph{WWTBAM?}) and \emph{Codecademy}. These three games(if \emph{Codecademy} counts as a game) make use of game elements in varying ways. While the focus is on analysing the paper itself and not necessarily the three games used it would be useful to at the very least mention how these three handle gamification of their respective activities.  

While \emph{Kahoot!} and \emph{Codecademy} primarily rely on pointsification elements like points, leaderboards and badges, the modified version of \emph{WWTBAM?} goes a little bit deeper. In comparison to \emph{Kahoot!} which shares the quiz like elements with \emph{WWTBAM?}, there is no point system that rewards the speed of answering. There is also a larger element of risk/reward in \emph{WWTBAM?} as there is a possibility of dropping to the previous score milestone if the wrong answer is given. If the players are happy with their current score, they can simply stop the game to keep their current points. Furthermore, there are the different lifelines which work as additional help that the players can use to make answering questions easier, albeit with limited usage. Due to this, we could categorise the use of \emph{WWTBAM?} as a trait of deep gamification while the other two are more shallow. 

When it comes to the course itself, it only makes use of pointsification elements through a public leaderboard to keep track of the total score between play sessions in \emph{Kahoot!} and \emph{WWTBAM?}. At a course level, this is purely a shallow gamification trait. Despite this, we should not discount that there is some added depth from the use of the three games and their integration into the course. 

\paragraph{Gamification in relation to learning outcomes}
The paper does place some thought into how the various games support the learning outcomes of the course. It is mentioned that \emph{Kahoot!} and \emph{WWTBAM?} are used as opportunities to quickly apply the knowledge that the students have recently learned and reinforce the learning outcomes in that manner. \emph{Codecademy} is then used so the students can practice their programming skills at their own pace. Since there are three relatively different components here that each are used as a means to support and reinforce the learning outcomes, we can categorise this as a trait of deep gamification.  

\paragraph{Gamification in relation to player types}
In terms of player types, there is little regard towards players who dislike competition. The use of a public leaderboard to tap into students natural competitiveness does not necessarily help those who fall behind or those who simply do not care for competition. A player that for example is at the bottom of the leaderboard might become demotivated towards improvement and give up instead. The choice of a public leaderboard does not necessarily help as a means to motivate struggling students either. The paper mentions in the conclusion that student engagement decreased during play sessions of \emph{Kahoot!} when students placed lower on the leaderboard, but it would have been nice to see an analysis on whether this was the case for the public leaderboard in the course as well. Furthermore, it would have been interesting to see comparisons between the more competitive aspect of \emph{Kahoot!} and the more cooperative aspect of \emph{WWTBAM?} as research suggests that cooperation can improve self efficacy~\cite{staiano2013adolescent} which in turn can increase motivation. 

As far as designing for player types is concerned this paper definitely demonstrates a trait of shallow gamification. 
Despite this, based on the results of the survey it seems like the students generally enjoyed the use of a competitive leaderboard which may suggest that the player types present in the sample were a good fit. 

\paragraph{Use of extrinsic and intrinsic motivators}
The study employs a few extrinsic motivators for the students throughout the course. These include the public leaderboard and winning candy or chocolate bars for placing at the top of leaderboards. A problem is that there is no explicit mention of how grading was handled for the experimental group, only how grading was handled prior to the gamification. Therefore it is hard to tell to which degree grades were used as extrinsic motivators in the course, but there is a mention of a final grade so some sort of assessment must have taken place. 

There is little discussion around intrinsic motivators, but it would not be stretch to assume that the three games provided some intrinsic motivation if the students actually enjoyed the activities. The paper mentions how students were intrinsically motivated by Codecademy's use of points and badges, but this falls at odds with their earlier definition of points and badges as extrinsic motivators which is relatively confusing. While it seems like the students seemed to enjoy the different approach to teaching, it is hard to say whether they had become more engaged due to the ''novelty factor'' of it or not. This is mentioned as a limitation in the conclusion. 

The paper also seems a bit light on the theoretical basis of motivation. Self-Determination Theory is brought up, but there is no mention of how the design of the course maps to the different components of the theory to support intrinsic motivation. Since the students seemed to enjoy the use of the three games in the course, we could at the very least assume that there was some intrinsic motivation involved and categorise this as being a component of deep gamification. 


\paragraph{Relationship towards failure}
It does not seem like this gamified course had any additional mechanics in place to support failure. The ambiguity around whether the total score of students on the public leaderboard resulted in the final grade or whether there was a final exam makes it hard to understand the assessment form of the course as well.  

\if{false}
\subsubsection{Editorial Analysis}
    * Title/abstract: seem fine enough. The abstract is pretty comprehensive and the title mentions one of the primary components of the study which is the leaderboard used. 
    * Problem statement:
        * Research questions: yes
        * The unit of study: seems clear
        * Relevant concepts: SDT is mentioned, related work as well. 
        * Research goal: Not explicitly mentioned
    * Structure and ease of reading
        * The initial paragraph in results would be better off in methods as it mentions the methods used to assess student engagement. 
        * Language is readable.
        * Relatively standard IMRAD structure otherwise. 
\fi

\subsubsection{Semantic Analysis}
\paragraph{Is the idea sound? Does it make sense?}
In general, there are many interesting ideas presented in this paper. The fact that average attention spans are taken into account when designing the lectures and splitting them into 20-minute segments to maximise the attention of students was intriguing to see. At the same time, one might wonder whether it is possible to cover the same amount of curriculum as a regular one hour lecture would. 

It is also good to see the use of three relatively different games as part of the course as it provides some variety which I have not seen much in other studies that make use of already existing solutions. The added variety may reduce the monotony that only using one game could result in.  

The main questionable component of the study is the use of a public leaderboard to rank the students according to their performance. While it seems to have worked for this specific experimental class it may not work as well for others with less competitive player types in the population. 

\paragraph{Does the paper contribute to the field of research?}
The paper can be seen as contributing to the field of research as it provides some interesting results that could be further explored by others. The relatively small sample size specifically warrants further research in the field using similar methodologies to the ones used in this paper. 

\paragraph{Do the results presented justify the conclusions?}
The presented results generally justify the conclusions that were given. The authors do not draw any major conclusions due to factors like sample size and the unknown effect of novelty factors. There is a mention that the findings suggest that the pedagogical goals in the introduction were met, despite there being no explicit mention of any pedagogical goals in the introduction. It can therefore be a bit hard to understand exactly what parts of the introduction they define as mentioning pedagogical goals.  

\paragraph{Does the solution justify the use of game elements?}
In general, it seems like the presented solution justifies the use of game elements. This is due to the fact that the entire course is built around the games and provides a more interactive experience for the students to participate in. One could argue that the mechanics of the three games could be extracted to create real-world counterparts, but this would cause the loss of instant feedback which is one of the primary components used for reinforcing what the students have been taught in the micro-lectures. It would also be more challenging to fit real-world counterparts to the games within 20-minute time slots as a less scripted experience might contain more overhead. 

\subsubsection{Research Analysis}
On the research end, the paper fares relatively well, but there are some missing components that would have benefited the research. 
The study used two different classes in consecutive years for sampling, the first as a control group of 54 students while the second year was used as an experimental group consisting of 52 students. These are relatively decent sample sizes, but there is no direct sampling method mentioned as the whole class is used in the study. This could be seen as a type of cohort sampling where the cohort consisted of students taking the course in the two years that the study was conducted. 

The use of research instruments includes observation of student behaviour, collection of administrative data about the students, an online survey as well as semi-structured interviews with focus groups on a small sample of students. The primary problem here is the lack of information for how the focus group was sampled which could result in cherry-picked responses. The sample size for the focus group is not provided and the questions asked during these interviews are left out as well. The modified version of \emph{WWTBAM?} which contains course relevant questions is not provided either, which in turn makes it hard to repeat the study. 

Identification of any variables that may affect the results is fairly limited. There is a mention of the fact that the authors are unsure whether or not the increase in engagement is due to novelty factors or not, but this is as far as the paper goes. Furthermore, since two different years were sampled it would have been nice to check whether the second year which experienced higher grades in the course did so simply because they already were more skilled than the initial control group or not. 

\subsubsection{Overall Assessment}
To summarise, the paper has a fair amount of traits which can be associated with deeper gamification, but also some which are shallow. There are a few interesting ideas at play in relation to the gamification of the course, but the theory side of the paper is somewhat lacking. There are also improvements that could be made, especially on the research side to make the study more repeatable and to increase validity. As far as quality is concerned, this seems to be a relatively decent study despite there being room for improvement. 




\subsection{Analysis of ''An interactive serious game via visualization of real life scenarios to learn programming concepts''}
\subsubsection{Shallow/Deep Gamification Analysis}
\paragraph{Use of ''surface level'' game mechanics}
The result of gamification in this paper is an actual serious game which consists of an array of various ''minigames''. Because of this, we will have to look a bit deeper into the game design than simply thinking about points, badges and leaderboards. As far as game mechanics are concerned, there are various minigames in the solution where each is intended to test the student's skills at a specific topic which has been presented to them beforehand. Because of this, there seems to be a fair amount of variety between the different minigames as far as teaching topics are concerned. The problem with these minigames is that they are too simple to conduct any useful analysis on. A fair amount of the minigames shown in the paper are very simple drag and drop type games or quizzes that have limited interaction with the player. The simplicity of the minigames does not necessarily create an interesting game experience from what the screenshots provided in the paper suggests. Since the minigames in general are too simple to be used for a more in-depth analysis on we could categorise this as a trait of shallow gamification or simply as inapplicable. 

\paragraph{Gamification in relation to learning outcomes}
The paper places an emphasis on a \emph{Game-Based Learning} methodology in the respect that each minigame should be aligned with the learning outcomes of the topic it is made for. The game component itself is primarily used as a means to reinforce the knowledge that the students have just been taught which in theory sounds good. Taking a look at the various minigame examples provided in the paper, we can also see that they at the very least are related to their respective topics. In general, as far as thinking about learning outcomes is concerned, this solution demonstrates a trait of deep gamification.  

\paragraph{Gamification in relation to player types}
The minigames presented in the paper are too simple to allow for categorisation of different player types as there is very little going in the actual games. While not directly related to player types, there is some thought put into customising the experience for players of various skill levels. A player of lower skill is presented with simpler theory and illustrations than their more adept counterparts based on how the players are scoring throughout the games. Regardless, it is hard to see that any additional effort has been put into tailoring the experience for a diversity of player types which is a trait of shallow gamification. 

\paragraph{Use of extrinsic and intrinsic motivators}
One of the biggest criticisms against this paper is specifically related to motivation and engagement. There is some mention of motivational theory and constructivism in the paper, but there are several cases of assumptions regarding the fact that a game automatically motivates and makes a player engaged because it is a game. An example of this is the following statement from the description of the system design: 
\begin{displayquote}
''Since, more interesting and engaging games are included as game activity, which makes learner to be more engaged in the game.''
\end{displayquote}

Outside of the grammatical problems of this sentence, it feels incorrect to assume that a game will motivate the player regardless of quality. There are several other cases where it seems like similar assumptions are taken which hurts the trust in the claims of the authors. Another example of this is from the conclusion of the paper:
\begin{displayquote}
''Even though, teaching concepts via games known to be a real challenge, here the learning strategy together with assessment strategy is implemented in the form of game; the learner will be more motivated to learn. It will be effective in demonstrating the concepts and engage the learner with concepts. Therefore, game-based learning enables a feasible learning environment that promotes intrinsic motivation.''
\end{displayquote}
There is a mention of intrinsic motivation here, but there is no description anywhere in the paper of how exactly the solution promotes intrinsic motivation or how it was developed with this in mind. As far as extrinsic motivators are involved, it is hard to say whether there were any incentives for taking part in the study or not as there was no mention of this. Regardless, it does not seem like this paper takes motivation into account while developing the solution which definitely is a trait of shallow gamification. 

\paragraph{Relationship towards failure}
As far as the relationship towards failure is concerned, the students are allowed to fail and retry the games. By failing, the student is brought back to the theory, displayed in a simpler manner before being allowed to try again. This sounds like a good idea, but the fact that the minigames are so simple can pose a problem as the students can simply brute force their way through these without really learning anything. 

\subsubsection{Semantic Analysis}
\paragraph{Is the idea sound? Does it make sense?}
The thought process of trying to use metaphors from the real life to teach programming concepts sounds good, but the problem lies with how the minigames handle this. 

To provide an example, there is a drag and drop minigame where the player is supposed to categorise various groceries into their respective bins as a way to teach about variable types. This would have made sense if the various bins represented things like fruits, vegetables and so on, but the minigame asks the player to place the groceries within integer, character and floating point bins which makes no sense as there is no real life translation from an apple to an integer. 
Furthermore, the idea of using real-world scenarios to teach programming concepts somewhat falls apart when the paper starts to introduce a space shooter minigame where the player has to shoot variables that are named incorrectly according to a naming convention.  

\paragraph{Does the paper contribute to the field of research?}
As far as contribution is concerned, it does not seem like this paper adds anything meaningful to the field of research. 

\paragraph{Do the results presented justify the conclusions?} 
The results that show off student performance between the experimental and the control group label the two as ''normal'' and ''Controlled'', making it hard to understand which is which. In general, data provided in the results to justify the claims that are provided in the conclusion. The reason for this is that the full list of answers in the survey is not provided, making the choice of extracted comments suspicious as they may have been cherry picked. 

\paragraph{Does the solution justify the use of game elements?}
To some extent, one could argue that the instant feedback and possibility of retrying to learn the concept better justifies the use of game elements. Despite this, a lot of the minigames presented seem like they could be textbook questions as the actual interaction with the minigames is a bit too simple. Due to the simplicity of most of the minigames, it could be possible to entirely replace these with online quizzes as the use of actual game mechanics is scarce. 

\subsubsection{Research Analysis}
The sample size in the study is fairly small in general, consisting of 20 students. There is a control group used, but no mention of how the students were distributed between control and experimental groups. The labelling for both groups is also very ambiguous as mentioned earlier. There is no mention of how the students were sampled and there are no attempts at identifying variables that could affect the results. A post-test questionnaire was used as a research instrument, but the questions in the survey and the full list of answers were not provided, making it hard to repeat the study. 

\subsubsection{Overall Assessment}
Overall, there is definitely a lack of quality throughout the paper in most aspects. Gamification of learning is handled in a fairly shallow manner and while the original idea of the paper is good, the execution is far too weak. 




\subsection{Analysis of ''Gamification in Computer Programming: Effects on Learning, Engagement, Self-Efficacy and Intrinsic Motivation''}
\subsubsection{Shallow/Deep Gamification Analysis}
\paragraph{Use of ''surface level'' game mechanics}
When it comes to the use of game mechanics for gamification, badges is the sole mechanic used. While we can consider pointsification in general as shallow gamification, there is still some inherent synergy when the three main components of points, badges and leaderboards are used in combination with each other. Only using one of these makes for an even shallower type of gamification. Badges in general can barely be defined as a game mechanic in isolation as they ultimately are a milestone that displays a person's progress at something. Furthermore, a problem with badges/achievements in the current day game industry is the oversaturation of these from doing trivial tasks. An example from the paper is that the students are awarded a badge for simply signing up for the web system that manages badges. By doing so, badges lose their value as milestones as they are supposed to represent an accomplishment, not trivial things. 

\paragraph{Gamification in relation to learning outcomes}
Given the shallowness of gamification involved, it hard to see any gamification that is directly in relation to any learning outcomes. Regardless, the paper mentions that the goal is to increase engagement as it is a factor that improves learning performance and indirectly useful for achieving the learning outcomes. 

\paragraph{Gamification in relation to player types}
Again, there is limited analysis possible here due to the simplicity of the gamification. In general though, this can be considered as not thinking of player types at all. 

\paragraph{Use of extrinsic and intrinsic motivators}
In regards to extrinsic and intrinsic motivators, we primarily see the use of extrinsic motivators in the form of badges as a tool to potentially improve engagement. There is also some additional discussion around Self-Determination Theory(SDT) and how badges relate to it that is useful to discuss here. 
The paper claims that the competence aspect of the SDT is pushed by badges as they give a sense of status recognition. This seems debatable as it is entirely possible for a player with fewer badges/achievements to be at a higher skill level than a player with more badges/achievements. As badges do not directly measure the skill level of the player, they are not really applicable to the competence aspect of the SDT. 

\paragraph{Relationship towards failure}
There is nothing that can be discussed in relation towards failure as the gamified solution too simple and shallow. 

\subsubsection{Semantic Analysis}
\paragraph{Is the idea sound? Does it make sense?}
As far as the idea is concerned, simply taking one of the most surface level mechanics from pointsification and applying it in isolation does not really feel or sound like a good or interesting idea. As mentioned in Section~\ref{sec:pointsification}, a game is an experience which is created from the various ways that many game mechanics are combined. What the study does in relation to gamification is so shallow that it is almost hard to call it either gamification or even pointsification as there are no additional game mechanics to synergise with. To some extent, even leaderboards and points can be seen as deeper game mechanics than badges as they have been components of games since the early days of computer games. 

\paragraph{Does the paper contribute to the field of research?}
When it comes to contribution, it could be considered that the study contributes by showing that badges in isolation do not have any significant use in relation to learning performance or motivation. While it is nice to have some research on the topic, it feels rather redundant as the concept is flawed from the start. 

\paragraph{Do the results presented justify the conclusions?} 
While the presented results could be seen as justifying the conclusion, the means of arriving at the given results are problematic. This is primarily in relation to the measurement of engagement which is mentioned to be significantly better in the experimental group. The paper uses the sum of badges earned from doing optional activities as the metric for player engagement. This is a rather flawed idea as it is entirely possible for a student to not do optional activities and still be very engaged with the course. If we draw a parallel to achievements in games, it is possible for a highly skilled and engaged player to not finish all of the achievements the game offers as they already are satisfied with the rest of the game. Sometimes, these optional activities are left out either becomes they simply offer no additional value to the player or simply because the player prioritises other things to spend time on. This does not mean that the player is any more or less engaged with the game though. 

Furthermore, the method for validating the completion of optional activities is somewhat questionable. This was handled by asking the students to upload a screenshot of their completion of the exercise from the learning management system. Simply uploading a screenshot is not necessarily a very good means of validating that the task actually was done as it can be easy to tamper with these. 

\paragraph{Does the solution justify the use of game elements?}
There is not really anything in this solution that justifies the use of game elements. Badges are simply added to an existing course as a means to keep track of how many optional exercises that the students have finished.  

\subsubsection{Research Analysis}
As far as sample sizes are concerned, there were 50 students in both the experimental and control group. There is no mention of how these were distributed though. Research instruments included a background survey before taking the subject, collection of grade point averages from the university and various pre- and post-test questionnaires. 
The paper does attempt to identify various variables that might affect the result, although it feels like there is a lack of background in games which may make the authors seem a bit naive in relation to what they think of as potential variables. It would have been nice to see a more introspective discussion around whether all the measurement methods were sufficient enough or whether there could be potential errors. The flawed assumption that engagement is increased by having more badges from optional activities might also explain why there is no positive effect on learning as the presented theory in the paper suggests. 
When it comes to repeatability, most of the methodology and research instruments have been well described. Despite this, there is limited information regarding the questions used for the various questionnaires. This is supposedly available upon request to the authors which could help to make the study repeatable assuming that the full data is provided. 

\subsubsection{Overall Assessment}
In general, the paper seems to be decent from an academic and research-oriented point of view, but there are major flaws in terms of the ideas and assumptions that really make it hard to see the value in the study. The gamification employed is so shallow that it almost is hard to actually call it a gamification or even pointsification as only one single mechanic actually was used. 







\subsection{Analysis of ''Game Based Learning - A Way To Stimulate Intrinsic Motivation''}
\subsubsection{Shallow/Deep Gamification Analysis}
\paragraph{Use of ''surface level'' game mechanics}
Generally, there is no use of any surface level pointsification mechanics in this paper as the focus is more on the game based learning angle of gamification. While there are no surface level game mechanics to be seen in the description of the course design, we can see that there is a focus on a somewhat deeper element of games which is player choice. This is handled by giving the students a lot of freedom in how they handle the various mandatory assignments in the course. The project can be anything as long as it is an educational game. Both competition and cooperation are also completely optional in terms of assignments and other parts of the course. Outside of providing the students with a fair amount of choice, there really is not that many other game mechanics that can be identified from the course design as the description is too high level and lacks a fair amount of details. 

\paragraph{Gamification in relation to learning outcomes}
While the paper is relatively high level in terms of the course design, it is possible to think that teaching students how to make games by actually making the games is to some extent related to the learning outcomes. It is somewhat hard to consider this as a gamification though as this is more of a ''learn by doing'' type approach that does not really rely on any game elements.

\paragraph{Gamification in relation to player types}
The paper does handle player types fairly well in the design of the course. It is not explicitly mentioned in the design, but the fact that competition and cooperation are optional provides a good amount of choice that should satisfy a wider range of player types. This is definitely a trait of deeper gamification. 

\paragraph{Use of extrinsic and intrinsic motivators}
When it comes to the use of extrinsic and intrinsic motivators, the course is primarily built upon the \emph{Taxonomy of Intrinsic Motivation}~\cite{maloneTax} where the purpose is to facilitate intrinsic motivation. There is some detailing on how the course is designed with the various components of this taxonomy in mind, but these primarily consists of short paragraphs for each component. It would have been preferable to see the paper go a little bit more in-depth and be more explicit in how the course is built around the taxonomy. This is due to the lack of detail which can make it a bit hard for other researchers to do similar things in the future. 
There is generally an attempt at avoiding extrinsic motivators as much as possible, but ultimately there is still a final grade, which counts as an extrinsic motivator. 

While it does not seem like the course can change anything about the extrinsic motivator that is the final grade, it does seem like it attempts to at the very least support intrinsic motivation through the use of the presented taxonomy. This can be seen as a trait of deeper gamification. 

Another point of discussion is related to the presented taxonomy which is the requirement of challenge to provide intrinsic motivation. While challenge absolutely can be an important motivational factor in games, there are also more narrative experiences released in recent years like \emph{The Walking Dead} series and most games in the visual novel genre. These games do not really rely on any sort of challenge as the focus is more on a narrative and cinematic experience which is closer to books and movies. Regardless, there is still a central element of player choice which leads the story in the game. As far as challenge is concerned we could think of this as the challenge of achieving the best possible ending, but it is ultimately not a focus of these types of games. As such, it could be beneficial to also look into what and why players enjoy these types of experiences and see if there is anything that could be used for gamification.

\paragraph{Relationship towards failure}
As far as supporting failure is concerned, there is no mention of anything related to this in the paper. 

\subsubsection{Semantic Analysis}
\paragraph{Is the idea sound? Does it make sense?}
The general idea of designing a course with intrinsic motivation in mind is fine, but it would have been nice to see a deeper discussion around the gamification of the course. Both gamification and game-based learning are mentioned background topics, but outside of the topic of the course, it is not really easy to see where the game elements are. 

At the same time, it feels like the presented taxonomy is handled in a very checklist type manner as the justification for some of the components is rather weak. Instead of forcing each component of the taxonomy into a gamification, it would probably be better to consider each individual component and see whether it is relevant or not for usage. 

\paragraph{Does the paper contribute to the field of research?}
The fact that parallel running courses with more extrinsic motivators may negatively affect courses with a focus on intrinsic motivators is some useful insight that could be carried on further. Despite this, the research side of the paper is fairly weak and does not really support the validity of the findings. 

\paragraph{Do the results presented justify the conclusions?} 
There is a fairly limited presentation of results as no actual data is provided. Some student responses and teacher observations are given, but the results of the various questionnaires should have been provided as it may seem like the provided responses are purely cherry picked. 

\paragraph{Does the solution justify the use of game elements?}
The paper discusses how stereotypical gamification can have negative consequences in the background section, but it is generally hard to find any game mechanics used in the design of the course. There may have been more game mechanics at play in the course, but the limited amount of detail provided by the paper makes this hard to see. Therefore it is somewhat hard to see whether the solution actually justifies the use of game elements as there does not really seem to be that big of a focus on them. 

\subsubsection{Research Analysis}
In general, the research aspect of the paper is fairly weak. There is a sample size of about 50 students which is similar to the other papers, but there is no use of any control group. While the paper does not really measure whether there is an increase in motivation from how the course is designed, this should probably have been part of the research. By doing so the paper could compare the motivation of a control class and an experimental class of students and potentially use the findings of this comparison to support the argument for using the presented taxonomy in future course designs. 

There is no mention of any sample methods used at all although it should be same to assume that all the students that took the course were part of the study. Research instruments consisted of course evaluation questionnaires and discussions with students and teachers. Since there was some data gathered, it is a shame to not see it properly presented. 
As far as variables are concerned there is to some extent some discussion around how other parallel courses can affect the results as the extrinsic motivators in those courses take priority. This is as far as any discussion around variables is taken though. 

The repeatability of the paper is rather weak as there is no data on questionnaire questions, answers or anything that could be used for future research. The general course design is also too high level to fully repeat the design in another study.  

\subsubsection{Overall Assessment}
Overall, there is a strong theoretical basis in this paper, but the description of the course design and its relation to the theory is too short. A deeper discussion would have been preferable as analysing the inner workings of the course is hard given the limited details. It is also hard to define whether this study performed deep or shallow gamification as there simply was too little information on the use of game mechanics despite there being some deeper aspects of gamification present.