\section{Introduction}
Serious games and gamifications are used for a wide array of topics with various purposes. Gamification in particular is handled in varying degrees of ''depth'' as some developers take the path of least resistance by adding checklist type game mechanics while others attempt to go deeper and create more interesting experiences. Despite this, there is somewhat of an oversaturation within the market in relation the more ''shallow'' form of gamification. This gives a relatively wrong impression of the capabilities of gamification as many developers only scratch the surface of possibilities. Furthermore, it also gives the term ''gamification'' a somewhat incorrect semantic meaning as many of the gamifications we can see today could rather be called ''pointsification'' as they only deal with points, badges and leaderboards. 

Due to these varying depths, it could be interesting to attempt making a distinction between what we consider as shallow and deep gamification. This is particularly useful in the area of education as gamification is primarily used as a means to increase motivation and potentially improve the learning outcomes of the students. By looking into deeper forms of gamification, it might be possible to find elements that increase intrinsic motivation in a better manner than the surface level as it usually relies on extrinsic motivators instead. 


This literature review will evaluate four different papers related to gamification and serious games for education, particularly in programming/game related courses. The focus of the review is to assess the depth of the gamification as well as checking the overall quality of the papers themselves. 