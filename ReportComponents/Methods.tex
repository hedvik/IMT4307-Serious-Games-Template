\section{Methods}
\subsection{Analysis Framework of Papers}    
Since game mechanics are applied at various depths in gamification it is necessary to first make a distinction between what we consider as ''shallow'' and ''deep'' gamification. 

\subsubsection{Pointsification}
To start of, we should define what shallow gamification is before moving deeper. A starting point would be ''Pointsification''~\cite{robertson2010} which is a term coined by Margaret Robertson. Pointsification primarily consists of using game elements like points, leaderboards and badges/achievements as a tool to motivate players into performing tasks. While there is nothing that is necessarily wrong with using these elements, it could be seen as a quick and simple means to add game elements to the gamified activity. These game elements are relatively universal and can be found in most games, but at the same time they are not the most essential components that make a good game interesting and fun to play. As such, pointsification elements could be seen as ''surface level'' components of games. Points, badges and leaderboards can be useful to augment the main activity by providing feedback on performance and progress, but do not necessarily make the main activity any more or less fun than it already is. It is important to consider that games are an experience which is the result of many various mechanics that are mixed together in different ways. The reductionist approach of simply taking some elements out of a game and applying them elsewhere might not give the same effect as they become more isolated and may lose their synergy with other mechanics. 

Furthermore, using pointsification as a sort of checklist to gamify an activity means that less thought is put into how these mechanics actually support the learning outcomes of the activity. Leaderboards as a game mechanic could for example support the learning outcomes of an activity if the goal was to teach the player about competition, but for most topics it simply ends up as a potential motivator for players that like competition. It may motivate some players to learn more, but also demotivate others.

\subsubsection{Player types}
An important consideration when gamifying an activity is that the use of game elements should complement the target player demographic. An example of shallow gamification would be to ignore this. To get an approximation of the different types of players we could look at the BrainHex~\cite{nacke2014brainhex} paper and their defined player types. It might not be a perfect approximation of player types as the the questionnaire used is relatively short and simple, but should provide some insight into what different types of players like and dislike. On the BrainHex website there is also a section on BrainHex Exceptions~\cite{brainhexexceptions} which are aspects of games that players dislike and their respective opposite classes. 

Two of these exceptions are the ''No Punishment''(Opposite of the Conqueror class) and the ''No Pressure''(Opposite of the Daredevil class) which could be relevant in relation to how player types react to pointsification. The ''No Punishment'' exception means that the player dislikes repetition of the same task which is a component of leaderboards. The ''No Pressure'' exception dislikes performing under pressure and time limited tasks which can be problematic for time based scoring systems and leaderboards. This means that if pointsification elements are used, they should be used in a demographic which consists of primarily Conquerors and Daredevils. In general though, it would be beneficial to consider that there may be players be demotivated by certain aspects of the gamification and provide alternatives if possible. 

\if{false}
While there is no data on the distribution of BrainHex exceptions, we can look at the distribution of classes where Conquerors and Daredevils are either primary or secondary class. 
            * 49.7\% of this sample includes Conqueror as either primary or secondary class.  
            * 14.9\% of this sample includes Daredevil as either primary or secondary class. 
            * While a sizable amount of their sample is Conquerors which may favour these pointsification mechanics, there is still a majority where the preference is unknown as the exceptions are independent of player types outside of their opposite. 
        * Regardless, there isn't any major conclusions to be drawn here as the game industry and market has changed a fair amount since 2010.
    * Some players can benefit from competition and leaderboard like gamification, others might become demotivated because they cannot do well or do not like to share their performance publicly. 
    * TLDR: Does the use of gamification support the player types that may be present in your demographic?
\fi

\subsubsection{The relationship towards failure}
An aspect of deeper gamification I would like to highlight is the relationship towards failure. This is primarily relevant towards gamification of education, but still a useful topic to mention regardless. Games generally allow the player to keep attempting as many times as they want until they are satisfied with their performance. There is often some punishment for failure, but the player is usually able to retry in order to overcome the challenge they face. This is contrary to how the real world works as students are required to perform well to get their grades, but unable to properly retry as many times as they want to improve. This lowered consequence of failure we see in games compared to the real world is useful to consider when gamifying education as it encourages the students to retry and improve if they are unsatisfied with their results. 
This opens up for a more trial and error type approach that potentially could lead to more knowledge transfer if the player is dedicated to improve. This approach is a central component of challenging games like \emph{Dark Souls} and \emph{Super Meat Boy} as it allows the player to hone their skills through repetition until they overcome the challenge they face. 

*TODO: CITE TRIAL AND ERROR IS GOOD?*

\subsubsection{Shallow And Deep Gamification}
We can thus define traits of shallow gamification in this way:
\begin{itemize}
    \item ''Surface level'' elements like points, badges and leaderboards are used without any other mechanics to synergise with.
    \item Little thought is put into how the game elements in the gamified activity support the learning outcomes. 
    \item The use of game elements does not support the player types present in the demographic of the gamified activity. 
\end{itemize}

As the contrast to shallow gamification, we can define traits of deep gamification as follows:
\begin{itemize}
    \item Going deeper than just using the ''surface level'' pointsification elements. Pointsification elements can be present, but they should be used in conjunction with other mechanics as well. 
    \item Thought is put into how the game elements in the gamified activity support the learning outcomes.
    \item The use of game elements supports the player types present in the demographic of the gamified activity.
    \item The gamified environment supports failure and allows the player to improve by retrying and optimising their performance. 
\end{itemize}


\subsection{Other Quality Assessment Criteria}
While the previous section described the analysis framework for making the distinction between shallow and deep gamification, this section goes through additional quality assessment criteria that will be used in the literature review. These criteria are based on what we used when assessing papers for the presentations in the serious games course with some additions. 

\subsubsection{Editorial Criteria}
On the editorial side, the following questions will be asked:
\begin{itemize}
    \item Does the title and abstract accurately reflect the content?
    \item Does the paper contain an explicit problem statement describing:
    \begin{itemize}
        \item Research questions (what do we want to know?)
        \item The unit of study (about what?)
        \item Relevant concepts (what do we know already?)
        \item Research goal (what do we expect to achieve?)
    \end{itemize}
\end{itemize}

\subsubsection{Semantic Criteria}
Furthermore, the semantic criteria are as follows: 
\begin{itemize}
    \item Is the idea sound? Does it make sense?
    \item Does the paper contribute to the field of research?
    \item Do the results presented justify the conclusions of the paper?
    \item Does the solution justify the use of game elements?
\end{itemize}

\subsubsection{Research Criteria}
The final list of criteria is related to the quality of the research. This includes: 
\begin{itemize}
    \item Are the drawn conclusions supported by the sample size?
    \item Does the study make use of control groups when measuring effects?
    \item What are the sample methods used? How does it affect the results?
    \item What research instruments were used for data collection?
    \item Does the paper attempt to identify variables that may affect the results?
    \item Are the results repeatable? If not, what has been left out?
\end{itemize}

\subsection{Literature Acquisition For The Review}
    * Google Scholar
        * "serious games" "education" "programming" "intrinsic" "extrinsic" "motivation" OR "motivators"
        * After 2014
        * 531 results
        * Papers from the first three pages were picked:
            * As long as they weren't a survey/literature review
                * Primarily because I want to focus on actual papers and their approaches to motivating players. 
            * As long as keywords in either title or abstract included "programming" and intrinsic/extrinsic motivations. The papers also had to be related to serious games for education. 
                * To make sure it is relevant to the topic
        
        * 2 papers were picked out for further reading based on the criteria. (Page 1)
        * 2 papers were picked out for further reading based on the criteria. (Page 2)
        * 0 papers were picked out for further reading based on the criteria. (Page 3)
        * 2 papers were set aside as potentially interesting to read in relation to the topic of motivations despite not including the term "programming" or being related to education 

    * IEEE Xplore
        * "serious games" "education" "programming" "intrinsic" "extrinsic" "motivation" OR "motivators"
        * 12 results by default
        * Papers were picked based on the same criteria as Google scholar.
        
        * 1 paper was picked out for further reading as it was the only one related to programming and intrinsic/extrinsic motivation keywords. 
        * 1 paper was set aside as potentially interesting to read in relation to the topic of motivations despite not including the term "programming" or being related to education 
        
        
    * From the 5 papers found, 1 was removed due to being written too poorly to be in a state for analysis
        * The english was simply unreadable that understanding the intent was impossible. 
    * 1 was also removed as it was not directly accessible
    * Leaving us with 4 papers + 1 used for the presentation. 