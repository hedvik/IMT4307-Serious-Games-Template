\section{Conclusion}
To summarise, the sample of papers included varying degrees of quality, both in terms of gamification depth and other aspects like research and semantics. There are definitely improvements that could be made to create more interesting and deep gamifications in all of the reviewed papers. The primary areas that could see improvements includes designing with various player types in mind as well as supporting failure instead of discouraging it.  

Given the small sample of reviewed papers and the narrow topic, it would have been interesting to see an extended review of gamification within the area of education. Extending the topic to gamification of education in general, could increase the amount of potential papers to review and also increase the chance of finding papers of higher quality. It would also be easier to review more papers with a slightly lighter framework that primarily focuses on the depth of gamification instead of a combination of gamification depth and paper quality. 