\section{Background}
The general background around gamification and serious games for education is that they often are used as a means to improve student motivation. The thought process behind this is that since games in general are seen as an engaging and motivating activity, it should be possible to take some elements from game design and apply them to other activities. The reason for actually attempting to improve motivation is because research suggests that motivation, particularly of the intrinsic variety is correlated with better learning outcomes~\cite{ryan2000intrinsic}.

\subsection{Relevant theories from the reviewed papers}
Since the reviewed papers make use of different motivational theories when designing their solutions, it could be useful to provide a short summary of what these theories revolve around.  

The first of these is Self-Determination Theory~\cite{ryan2000intrinsic}. The Self-Determination Theory describes three basic needs that generally are associated with intrinsic motivation. These include Agency/Autonomy, Competence and Relatedness. Agency/Autonomy is related to providing choice about the events that happen and being able to control the current situation, Competence is about being rewarded for the achievement of challenging tasks, and Relatedness is that actions matter in the wider society~\cite{simonSDTshort}. Satisfying these three needs is often what many serious games and gamifications are based on as a means to increase the intrinsic motivation of students. 
    
There is also the Taxonomy of Intrinsic motivation~\cite{maloneTax} which is an extension of Malone's motivational model~\cite{malone1981toward}. In this taxonomy, there is a distinction between two types of intrinsic motivation. The first of these is internal motivation which is more focused on an individual's personal goals. Internal motivation includes the following components~\cite{maloneTax, mozelius2014game}:

\begin{itemize}
    \item Challenge in relation to goals, uncertain outcomes and performance feedback
    \item Curiosity in relation to sensory and cognitive inquisitiveness
    \item Control in relation to contingency, choice and power
    \item Fantasy in relation to interwoven emotional and cognitive aspects
\end{itemize}

Interpersonal motivation on the other hand is more focused around the relations between individuals and how it affects motivation. Interpersonal motivation includes the following components~\cite{maloneTax, mozelius2014game}:

\begin{itemize}
    \item Cooperation in relation to players working together to achieve goals
    \item Competition where players compete against each other to achieve goals
    \item Recognition in relation to making achievements available for others
\end{itemize}

These various components of internal and interpersonal motivation can then be considered when developing serious games and gamifications as a means to improve intrinsic motivation.