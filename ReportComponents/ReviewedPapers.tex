\section{Reviewed papers}
\subsection{Climbing up the leaderboard: An empirical study of applying gamification techniques to a computer programming class}
''Climbing up the Leaderboard'' by Panagiotis et al.~\cite{panagiotis2016climbing} is a paper that researches whether motivation can be improved in an entry-level programming course using gamification. The gamification of the course is handled by using a mixture of \emph{Kahoot!}, a modified version of \emph{Who Wants To Be A Millionaire?} and \emph{Codecademy} to create a more interactive experience for the students to partake in. A public leaderboard was also used which contained score totals for students between play sessions of the various games. The general results of the study showed that the students liked this approach to teaching and that there was an increase in motivation. 
 
\subsection{An interactive serious game via visualization of real life scenarios to learn programming concepts}
''An interactive serious game via visualization of real-life scenarios to learn programming concepts'' by Sajana, Kamal \& Jayakrishnan~\cite{sajana2015interactive} is a paper that studies gamification of an entry-level programming course by using a game-based learning methodology. The course itself is structured so each topic is a self-contained module. The result of gamification is an array of various mini-games that are used at the end of each module to reinforce what the students recently have been taught. The teaching of topics is handled by finding real-life scenarios and using these as metaphors for programming concepts. The provided results seem to point out that the students enjoyed the gamification and that there was an increase in motivation. 

\subsection{Gamification in Computer Programming: Effects on Learning, Engagement, Self-Efficacy and Intrinsic Motivation}
''Gamification in Computer Programming: Effects on Learning, Engagement, Self-Efficacy and Intrinsic Motivation'' by Ortiz-Rojas, Chiluiza \& Valcke~\cite{ortiz2017gamification} also checks how gamification affects students of an entry-level programming course. In this study, gamification is performed by only using badges as there are limited or no studies on the effect of individual gamification components. Badges are given to the students for completing various tasks throughout the course and there are additional badges that can be gained from optional work. The engagement of students is measured by how many optional badges they can achieve. The provided results point out that badges increased student engagement, but not learning performance. 

\subsection{Game Based Learning - a way to stimulate intrinsic motivation} 
''Game Based Learning - a way to stimulate intrinsic motivation'' by Mozelius~\cite{mozelius2014game} is a paper that describes the overarching design of a course in developing educational games using game-based learning. The course was built with the Taxonomy of Intrinsic Motivation~\cite{maloneTax} in mind as a means to increase the intrinsic motivation of students. The general findings of the research are that student motivation increased, but parallel courses with more extrinsic motivators ended up getting priority which undermines the intrinsic motivation. 